\begin{figure}[H]
	    	\centering
	    	\fbox{\includegraphics[width=0.5\textwidth]{MEAN}}
	    	\caption{MEAN.js logo}
	    	\label{fig:MEAN}
   	\end{figure}
The installation is quite an easy process as Crowd Source uses standard web technologies. We can deploy our system in one of two ways. One way is to use a docker image to load the system the other is to set it up by loading all the dependancies to a machine. Crowd Source is using the mean stack so should you encounter any issues please consult their webpage. \url{http://mean.io}. The github readme also houses the installation instructions.
\subsection{Setting up the environment}
To use our application the following prerequisites need to be met. Please note that installing packages globally requires super user privillages on unix based systems and administrator privileges on windows.
\begin{itemize}
	\item git. It is important to have a good understanding of this technology. It must be installed on your system for development.
	\item NPM(Node Package manager) and node.js
	For the latest information on installing npm please visit \url{https://nodejs.org/download/} Once you have this setup properly it is important to make sure you have the latest version installed.
	\item BOWER for client side dependancies.
	This technologie works with our stack and the latest version can be installed by making use of npm.
	\newline
	npm install -g bower
	\item MongoDB document database. Mongo has different installation instructions based on the operating system being used thus it is best to consult the following website \url{https://www.mongodb.org} for the instructions.
	\item GRUNT task manager. GRUNT can be installed through npm. 
	\newline
	npm install -g grunt-cli
\end{itemize}
Crowd Source is now ready to be started. First make sure that Mongodb is up and running. Next clone the project form gitHub by running the following command.
\newline
git clone \url{https://github.com/FrikkieSnyman/COS301_GroupProject.git}
\newline
Install the dependancies by running the following command.
\newline
npm install
\newline
Running the grunt command in the root directory of the project will start the server. The server will start in the development environment on port 3000 by default. Crowd Source currently supports running in three environment types.
\begin{itemize}
	\item{Development} is used to run the server for development, non minified files are used so that errors can be properly shown.
	\item{Production} loads all the minified sources and also only non development dependancies. This is the method to use when running the server for commercial use.
	\item{Secure} is just a enhanced version of production with added ssl security. For this environment to work it requires the necessary certificates.
\end{itemize}
After this the server should be functioning and you can move on to getting started. Check your favourite web browser and the server should be running on port 3000.
\subsection{Using docker}
This is still under development and will be implemented soon.
\url{http://docs.docker.com/mac/started/}
\subsection{Deploying on your server}
Changing environment setting to suite your current need is quite easy. Under your project directory there is a config folder. It contains all the variable that can be changed for the various services. Using passport.js for authentication allows you to integrate with the various services for authentication. We currently support Facebook, Google and Twitter authentication. Please visit \url{http://passportjs.org} for further instructions on how to configure this services. We recommend using digital ocean servers to run this project on. \url{https://www.digitalocean.com}

\subsection{Installation failures}
Should the installation fail please make sure that all the MEAN.js dependancies are installed by visiting their website. If the issue cannot be resolved please log a issue on gitHub.