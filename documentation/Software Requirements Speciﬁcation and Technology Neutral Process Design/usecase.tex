%============================
%User Management
%============================
\subsection{User Management}
This module is responsible for reading data from the client user database. It will do this by making use of ldap.js .
\subsubsection{Scope}
\paragraph{Test}
\subsubsection{Use cases}
\begin{itemize}
\item Autorize
\item validateUserName
\item retrieveEmail
\end{itemize}
\subsubsection{Domain model}

%============================
%Project
%============================
\subsection{Project}
The project module is responsible for the representation and persistence of all projects that the system will use to do the estimations on. This module will allow for complex projects to be created, as well as to be updated. A project is represented as a tree, consisting of a top-level project node and lower-level task nodes.

\subsubsection{Scope}

\subsubsection{Use cases}

\paragraph{Create a project - priority: critical}
Users with sufficient privileges can create projects. This will persist the project to the database.

\paragraph{Get project - priority: critical}
Users with sufficient privileges can retrieve projects to view them, or to be used for other purposes. This use case must thus return a representation of the project-tree, and not produce the output of the project tree.

\paragraph{Hide project}
Projects can be effectively deleted by hiding them, seeing as projects are not physically removed from the database. This action can only be done by authourised user.

\paragraph{Update project - priority: critical}
This enables users to update projects, if they have sufficient privileges, by adding or removing task nodes to and from the project tree.

\subsubsection{Domain model}

%============================
%Estimation
%============================
\subsection{Estimation}
	The Estimation module will be responsible for handling all of the actions related to the estimation of a project. This involves allowing a user to place an estimate on a specific task as well as calculating the total estimation of a project.
\subsubsection{Scope}
	This is the estimation scope
	\begin{figure}[H]
	    	\centering
	    	\fbox{\includegraphics[width=0.5\textwidth]{Estimation_Scope}}
	    	\caption{Estimation Scope}
	    	\label{fig:Estimation_Scope}
   	\end{figure}
\subsubsection{Use cases}
	\paragraph{estimateTask - priority:critical}This system will allow a user to place an estimation on a task.
	\paragraph{calculateTotalEstimation - priority:critical}This system will traverse the project tree and calculate the total estimation of a project.
\subsubsection{Domain model}
%============================
%Report
%============================
\subsection{Report}
The Report module will do statistical analysis on the data as well as draw graphs of the estimation data. This module will be responsible for any analysis of the data. It's functionality will be expanded as development continues.
\subsubsection{Scope}
The scope of this module will be accessible to any user at this point in the development, the only authorization that will take place is the initial log in of the user.
	\begin{figure}[H]
	    	\centering
	    	\fbox{\includegraphics[width=0.8\textwidth]{Report_Scope}}
	    	\caption{Report Scope}
	    	\label{fig:Report_Scope.png}
   	\end{figure}
\subsubsection{Use cases}
	\paragraph{Statistical Analysis: critical}
	User accessing this function will receive statistical analysis of the data for the chosen project.

	\paragraph{Generate Graphs: critical}
	User accessing this function will receive a series of graphs of the data for the chosen project. 
\subsubsection{Domain model}

%============================
%Notification
%============================
\subsection{Notification}
This module will be responsible to notify all the users as required by the projects estimation. The estimation module will log a message request to the notification and add all the required users to notify.
\subsubsection{Scope}
This is the notification scope
	\begin{figure}[H]
	    	\centering
	    	\fbox{\includegraphics[width=0.8\textwidth]{Notification_Scope}}
	    	\caption{Notifications Scope}
	    	\label{fig:Notification_Scope}
   	\end{figure}
\subsubsection{Use cases}
The main purpose of notifications is to notify users with the required message. The estimation module will supply the list of users to notify and the contents of the message.
\paragraph{sendMesssage -- priority:niceToHave}
\paragraph{createNotification -- priority:niceToHave}
\subsubsection{Domain model}
